\documentclass{article}
\usepackage{a4}
\usepackage{listings}
\lstset{
	columns=fullflexible,
	frame=single,
	breaklines=true,
}
\title{Protecting democracy with a trustless blockchain based decentralised election system}
\author{Sumuk Shashidhar}
\date{Janurary 2021}

\begin{document}
	\maketitle
	\begin{abstract}
		Democracy fades as sophisticated attempts of voterfraud are detected, with some even succeeding. VoteBlock attempts to protect democracy by decentralising the election process to ensure the lack of a single point of failure or control, with the help of a blockchain. It must be understood that while VoteBlock secures the election process, it does not secure the voter registration process essential for authorizing each voter. 
	\end{abstract}
	\pagebreak
	\tableofcontents
	\pagebreak
	\section{Introduction}
	\section{Structure of A Block}
	\subsection{Code Used}
	\textbf{block.py}
	\begin{lstlisting}[language=Python]
from hashlib import sha256
import json

class Block:
	def __init__(self, index, transactions, timestamp, previous_hash, nonce=0):
			"""This function initialises the block of the blockchain using the regular concept
			
			Args:
			index ([type]): Index number of the block
			transactions ([type]): The Transactions to be stored in the given block
			timestamp ([type]): The timestamp of the given block
			previous_hash ([type]): The hash of the previous block to store
			nonce (int, optional): The number only used once. The nonce is to maintain uniqueness, making it hard to regenerate, which gives the blockchain the power it needs. Defaults to 0.
			"""
			self.index = index
			self.transactions = transactions
			self.timestamp = timestamp
			self.previous_hash = previous_hash
			self.nonce = nonce

	def compute_hash(self):
		"""
		A function that return the hash of the block contents.
		"""
		block_string = json.dumps(self.__dict__, sort_keys=True)
		return sha256(block_string.encode()).hexdigest()
	\end{lstlisting}
	
\end{document}
