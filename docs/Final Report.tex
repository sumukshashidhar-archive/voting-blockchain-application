\documentclass{article}
\usepackage{a4}
\usepackage{listings}
\lstset{
	columns=fullflexible,
	frame=single,
	breaklines=true,
}
\title{Protecting democracy with a trustless blockchain based decentralised election system}
\author{Sumuk Shashidhar}
\date{\today}

\begin{document}
	\maketitle
	\begin{abstract}
		Democracy fades as sophisticated attempts of voterfraud are detected, with some even succeeding. VoteBlock attempts to protect democracy by decentralising the election process to ensure the lack of a single point of failure or control, with the help of a blockchain. It must be understood that while VoteBlock secures the election process, it does not secure the voter registration process essential for authorizing each voter. 
	\end{abstract}
	\pagebreak
	\tableofcontents
	\pagebreak
	\section{Introduction}
	\section{Structure of A Block}
	\subsection{Components}
	Each block consists of the following elements essential to recording votes and identifying induviduality of each voter.
	\begin{enumerate}
		\item Index - To enumerate each block
		\item Transactions - The given transactions in each block (votes)
		\item Timestamp - The timestamp of each block's creation
		\item Previous Hash - The hash of the previous block to facilitate the blockchain
		\item Nonce - The number only used once. Used to supplement the rest of the data to generate a desired hash pattern
	\end{enumerate}
	\subsection{Hashing}
	SHA-256 hashing is used, where the data of the block is first dumped into a string, and then computed after being unicode encoded.
	\subsection{Code Used}
	\textbf{block.py}
	\begin{lstlisting}[language=Python]
from hashlib import sha256
import json

class Block:
	def __init__(self, index, transactions, timestamp, previous_hash, nonce=0):
			self.index = index
			self.transactions = transactions
			self.timestamp = timestamp
			self.previous_hash = previous_hash
			self.nonce = nonce

	def compute_hash(self):
		block_string = json.dumps(self.__dict__, sort_keys=True)
		return sha256(block_string.encode()).hexdigest()
	\end{lstlisting}

\section{The Blockchain}
\subsection{Components}
\subsubsection{Basic Structuring}
The Blockchain consists of a two objects. 
\begin{enumerate}
	\item Unconfirmed Transactions - To store a list of unconfirmed transactions.
	\item Chain - To store the chain data.
\end{enumerate}
\paragraph{Difficulty}
The difficulty is a simple integer that determines how hard it is to mine a block. The higher this integer is set, the more difficult it is to mine. 

\subparagraph{Method}
This is accomplished by using the number of leading zeros to the hash. A difficulty of 2 will ensure that there are two leading zeros for each accepted hash.
\subsubsection{Genesis Block}
The genesis block of any block chain is the first mined block of that blockchain. Here, we mine it with a list of empty transactions and null (0) data, and we add it to the chain.
\subsubsection{Last Block Property}
A useful property that is used to retrieve the last block added to the blockchain.
\subsubsection{Add Block Methodology}
It is very simple to add a block to the blockchain. It requires only the block object and the proof of work.
\paragraph{Verification}
includes: 
\begin{itemize}
	\item Checking if the proof is valid.
	\item The previous hash referred in the block and the hash of latest block in the chain match.
\end{itemize}

\subsection{Code Used}
\begin{lstlisting}[language=Python]
from block import Block
import time

class Blockchain:
	difficulty = 2
	
	def __init__(self):
		self.unconfirmed_transactions = []
		self.chain = []
	
	def create_genesis_block(self):
		genesis_block = Block(0, [], 0, "0")
		genesis_block.hash = genesis_block.compute_hash()
		self.chain.append(genesis_block)
	
	@property
	def last_block(self):
		return self.chain[-1]
	
	def add_block(self, block, proof):
		previous_hash = self.last_block.hash
		
		if previous_hash != block.previous_hash:
			return False
		
		if not Blockchain.is_valid_proof(block, proof):
			return False
		
		block.hash = proof
		self.chain.append(block)
		return True
	
	@staticmethod
	def proof_of_work(block):
	"""
	Function that tries different values of nonce to get a hash
	that satisfies our difficulty criteria.
	"""
	block.nonce = 0
	
	computed_hash = block.compute_hash()
	while not computed_hash.startswith('0' * Blockchain.difficulty):
	block.nonce += 1
	computed_hash = block.compute_hash()
	
	return computed_hash
	
	def add_new_transaction(self, transaction):
	self.unconfirmed_transactions.append(transaction)
	
	@classmethod
	def is_valid_proof(cls, block, block_hash):
	"""
	Check if block_hash is valid hash of block and satisfies
	the difficulty criteria.
	"""
	return (block_hash.startswith('0' * Blockchain.difficulty) and
	block_hash == block.compute_hash())
	
	@classmethod
	def check_chain_validity(cls, chain):
	result = True
	previous_hash = "0"
	
	for block in chain:
	block_hash = block.hash
	# remove the hash field to recompute the hash again
	# using `compute_hash` method.
	delattr(block, "hash")
	
	if not cls.is_valid_proof(block, block_hash) or \
	previous_hash != block.previous_hash:
	result = False
	break
	
	block.hash, previous_hash = block_hash, block_hash
	
	return result
	
	def mine(self):
	"""
	This function serves as an interface to add the pending
	transactions to the blockchain by adding them to the block
	and figuring out Proof Of Work.
	"""
	if not self.unconfirmed_transactions:
	return False
	
	last_block = self.last_block
	
	new_block = Block(index=last_block.index + 1,
	transactions=self.unconfirmed_transactions,
	timestamp=time.time(),
	previous_hash=last_block.hash)
	
	proof = self.proof_of_work(new_block)
	self.add_block(new_block, proof)
	
	self.unconfirmed_transactions = []
	return True
\end{lstlisting}
	
\end{document}
